\documentclass[10pt]{article}
\usepackage{times}
\usepackage{graphicx}
\title{Healthcare-ML: Clinical Risk Modeling with API Data}
\author{Jibran Kazi (JK)}
\date{February 2026}
\begin{document}
\maketitle
\begin{abstract}
We present a reproducible, API-driven clinical risk pipeline that trains a Random Forest
classifier on tabular healthcare data and evaluates it using AUROC, AUPRC, and calibration
analysis.  The model achieves near-perfect discrimination (AUROC $\approx 0.999$) on the
Wisconsin Breast Cancer dataset, but calibration analysis reveals that the predicted
probabilities are poorly calibrated---motivating future work on calibration-aware ensembles
for clinical deployment.
\end{abstract}

\section{Data \& Method}
We use the Wisconsin Breast Cancer dataset (569 patients, 30 continuous features derived
from fine needle aspirate images of breast masses).  The binary target indicates whether a
tumor is malignant (class~0) or benign (class~1).  Data is split 80/20 with stratification
(seed 123).

The pipeline applies \texttt{StandardScaler} normalization followed by a Random Forest
classifier (300 trees, unlimited depth, seed 42).  No hyperparameter tuning is performed;
this serves as a reproducible baseline.

\section{Results}
%% Auto-generated -- do not edit by hand
\begin{tabular}{l r}
\hline
Metric & Value \\ \hline
AUROC & 0.9987 \\
AUPRC & 0.9992 \\ \hline
\end{tabular}


\begin{figure}[h]
\centering
\includegraphics[width=0.45\linewidth]{figures/roc.png}
\includegraphics[width=0.45\linewidth]{figures/pr.png}
\caption{ROC and PR curves.  Both curves hug the top-left and top-right corners
respectively, indicating near-perfect discrimination between malignant and benign tumors.}
\end{figure}

\begin{figure}[h]
\centering
\includegraphics[width=0.6\linewidth]{figures/calibration.png}
\caption{Calibration curve.  The S-shaped deviation from the diagonal shows that the
Random Forest is overconfident: it pushes predicted probabilities toward 0 and 1 rather
than producing well-calibrated risk estimates.}
\end{figure}

\section{Discussion}
The model achieves AUROC of 0.999 and AUPRC of 0.999, demonstrating that a standard
Random Forest can nearly perfectly separate malignant from benign tumors on this dataset.
The ROC curve reaches $\sim$97\% true positive rate at essentially 0\% false positive rate,
and precision remains at 1.0 across nearly the full recall range.

However, the calibration plot tells a different story.  The predicted probabilities exhibit
a steep S-shaped curve rather than following the ideal diagonal.  For predictions around
0.4, the actual fraction of positive cases is near zero, while by 0.5 it jumps to
$\sim$60\%.  This means a clinician could not interpret a raw prediction of 0.7 as a
``70\% chance of malignancy''---the model's probability outputs are not trustworthy for
risk communication, even though its ranking ability is excellent.

This is a well-documented property of Random Forests: because predictions are averages of
tree votes, they tend to cluster away from the extremes but still produce overconfident
outputs.  Addressing this limitation through probability calibration (e.g.,
\texttt{CalibratedClassifierCV} with isotonic or Platt scaling) is the natural next step
and a core direction of this research program.

\section{Conclusion}
This pipeline demonstrates that high discrimination alone is insufficient for clinical
risk models.  While the Random Forest baseline achieves near-perfect AUROC/AUPRC, the
calibration analysis reveals that predicted probabilities require post-hoc correction
before they can be used for patient-facing risk communication.  Future work will
integrate calibration-aware training and evaluate on external clinical datasets.

\end{document}
