\documentclass[10pt]{article}
\usepackage{times}
\usepackage{graphicx}
\usepackage{booktabs}
\usepackage[margin=1in]{geometry}
\usepackage{hyperref}

\title{Healthcare-ML: Calibration-Aware Clinical Risk Modeling\\with Multi-Model Comparison}
\author{Jibran Kazi}
\date{February 2026}

\begin{document}
\maketitle

\begin{abstract}
We present a reproducible clinical risk pipeline that compares five classifiers---Logistic
Regression, Random Forest, Random Forest with isotonic calibration, Random Forest with
Platt scaling, and XGBoost---on the Wisconsin Breast Cancer dataset.  All models achieve
near-perfect discrimination (AUROC $> 0.98$), but calibration analysis reveals significant
differences in predicted probability reliability.  We show that post-hoc calibration via
\texttt{CalibratedClassifierCV} substantially improves the Brier score and calibration
curve alignment, confirming that high AUROC alone is insufficient for clinical deployment.
Bootstrap confidence intervals and SHAP-based feature importance provide statistical rigour
and interpretability.
\end{abstract}

\section{Introduction}
Clinical risk models must satisfy two requirements: (1)~accurate discrimination between
outcomes, and (2)~well-calibrated probability estimates that clinicians can communicate to
patients~\cite{niculescu2005}.  A model that ranks patients correctly but assigns a
predicted probability of 0.7 when the true event rate is 0.3 is dangerous in practice,
even if its AUROC is high.

Random Forests are popular in clinical ML for their strong discrimination, but their
predicted probabilities are known to be poorly calibrated~\cite{niculescu2005}.
Post-hoc calibration methods---Platt scaling~\cite{platt1999} and isotonic
regression~\cite{zadrozny2002}---can improve probability reliability.  Modern gradient
boosting methods like XGBoost~\cite{chen2016} offer an alternative with potentially
better-calibrated outputs.

This work systematically compares these approaches on a standard clinical dataset,
evaluating discrimination (AUROC, AUPRC), calibration (Brier score, reliability diagrams),
and interpretability (SHAP feature importance).

\section{Data \& Method}

\subsection{Dataset}
We use the Wisconsin Breast Cancer dataset (569 patients, 30 continuous features derived
from fine needle aspirate images of breast masses).  The binary target indicates whether a
tumour is malignant (class~0) or benign (class~1).  Data is split 80/20 with stratification
(seed~123), yielding 455 training and 114 test samples.

\subsection{Models}
Five classifiers are compared, all preceded by \texttt{StandardScaler}:
\begin{enumerate}
  \item \textbf{Logistic Regression} --- L2-regularised baseline (max\_iter=1000).
  \item \textbf{Random Forest (RF)} --- 300 trees, unlimited depth.
  \item \textbf{RF + Isotonic} --- RF wrapped in \texttt{CalibratedClassifierCV}
        with isotonic regression (3-fold internal CV).
  \item \textbf{RF + Platt} --- RF wrapped in \texttt{CalibratedClassifierCV}
        with sigmoid (Platt) scaling (3-fold internal CV).
  \item \textbf{XGBoost} --- 300 rounds, max\_depth=6, learning rate 0.1.
\end{enumerate}

\subsection{Evaluation Protocol}
All models are evaluated with 5-fold stratified cross-validation on the training set.
Final models are refit on the full training set and evaluated on the held-out test set.

Metrics include AUROC, AUPRC, and Brier score.  95\% confidence intervals are computed
via percentile bootstrap (1000 resamples).  SHAP TreeExplainer provides feature-level
interpretability for the best-performing model.

\section{Results}
%% Auto-generated -- do not edit by hand
\begin{tabular}{l r}
\hline
Metric & Value \\ \hline
AUROC & 0.9987 \\
AUPRC & 0.9992 \\ \hline
\end{tabular}


\begin{figure}[h]
\centering
\includegraphics[width=0.48\linewidth]{figures/roc.png}
\includegraphics[width=0.48\linewidth]{figures/pr.png}
\caption{ROC (left) and Precision-Recall (right) curves for all five models.
All achieve near-perfect discrimination, with overlapping curves above AUROC~0.98.}
\end{figure}

\begin{figure}[h]
\centering
\includegraphics[width=0.6\linewidth]{figures/calibration.png}
\caption{Calibration curves.  The uncalibrated RF shows characteristic S-shaped
overconfidence.  Both isotonic and Platt calibration substantially improve alignment
with the diagonal, as reflected in lower Brier scores.  Logistic Regression and
XGBoost are inherently better calibrated.}
\end{figure}

\begin{figure}[h]
\centering
\includegraphics[width=0.7\linewidth]{figures/shap_bar.png}
\caption{Top~15 features by mean absolute SHAP value for the best model, showing
which cell-nucleus measurements most influence the malignancy prediction.}
\end{figure}

\section{Discussion}
All five models achieve AUROC above 0.98, confirming that the Wisconsin Breast Cancer
dataset is highly separable.  However, the calibration analysis reveals important
differences.  The uncalibrated Random Forest exhibits the well-documented S-shaped
calibration curve~\cite{niculescu2005}, pushing predictions toward 0 and 1.

Post-hoc calibration via isotonic regression and Platt scaling both improve the Brier
score.  Isotonic calibration is more flexible and typically achieves better calibration
on this dataset, though it can overfit with very small samples.  Platt scaling, being
parametric, is more stable but assumes a sigmoidal relationship.

Logistic Regression, despite lower discrimination, naturally produces well-calibrated
probabilities because it directly models log-odds.  XGBoost offers a middle ground:
strong discrimination with reasonable calibration.

SHAP analysis reveals that features related to cell concavity, perimeter, and radius
are the strongest predictors---consistent with clinical knowledge that malignant tumours
tend to have larger, more irregularly shaped cells.

\subsection{Limitations}
This study uses a single, clean, moderately-sized dataset.  Real-world clinical data
involves missing values, temporal drift, and multi-site heterogeneity.  Future work
should evaluate on larger, noisier datasets (e.g., MIMIC-III) and investigate
calibration stability under distribution shift.

\section{Conclusion}
High AUROC alone is insufficient for clinical risk models.  This work demonstrates
that post-hoc calibration can bridge the gap between strong discrimination and reliable
probability estimates.  The complete pipeline---from data loading through cross-validated
comparison, bootstrap inference, SHAP interpretability, and automated LaTeX
reporting---serves as a reproducible template for clinical ML research.

\begin{thebibliography}{9}
\bibitem{niculescu2005}
Niculescu-Mizil, A.\ \& Caruana, R.\ (2005).
\textit{Predicting Good Probabilities with Supervised Learning}.
Proceedings of ICML, pp.~625--632.

\bibitem{platt1999}
Platt, J.\ (1999).
\textit{Probabilistic Outputs for Support Vector Machines and Comparisons to Regularized
Likelihood Methods}.
Advances in Large Margin Classifiers, pp.~61--74.

\bibitem{zadrozny2002}
Zadrozny, B.\ \& Elkan, C.\ (2002).
\textit{Transforming Classifier Scores into Accurate Multiclass Probability Estimates}.
Proceedings of KDD, pp.~694--699.

\bibitem{chen2016}
Chen, T.\ \& Guestrin, C.\ (2016).
\textit{XGBoost: A Scalable Tree Boosting System}.
Proceedings of KDD, pp.~785--794.

\bibitem{guo2017}
Guo, C.\ et al.\ (2017).
\textit{On Calibration of Modern Neural Networks}.
Proceedings of ICML, pp.~1321--1330.

\bibitem{lundberg2017}
Lundberg, S.\ \& Lee, S.-I.\ (2017).
\textit{A Unified Approach to Interpreting Model Predictions}.
Proceedings of NeurIPS, pp.~4765--4774.
\end{thebibliography}

\end{document}
